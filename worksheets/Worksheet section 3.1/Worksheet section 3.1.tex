\documentclass[11pt]{exam}
\usepackage[margin=1in]{geometry}
\pagestyle{plain}
\usepackage{amsmath,amsfonts,amssymb,amsthm,enumerate}
\usepackage{multicol}
\usepackage[]{graphicx}
\usepackage{hyperref}
\usepackage{tikz}
\usepackage{pgfplots}
\usepackage{subfigure}
\usepackage[final]{pdfpages}

\addtolength{\footskip}{2\baselineskip} % to lower the page numbers
\title{\vspace{-0.5in} Math 115 \\ Worksheet Section 3.1}
\date{}


% \theoremstyle{definition}
% \newtheorem{problem}{Problem}
\renewcommand{\questionlabel}{\textbf{Problem~\thequestion.}}
%\printanswers

\begin{document}
\maketitle
\vspace{-0.75in}
\begin{questions}
\question Find the derivatives of the following functions
  \begin{parts}
  \part \(a(x)=x^{12}\)
  \part \(b(x)=x^{3/4}\)
  \part \(c(x)=x^{-3/4}\)
  \part \(d(x)= \ln e^{ax}\) for \(a\) a constant
  \part \(e(x) = \sqrt{x}(x+1)\)
  \part \(f(x) = \frac{x^2+1}{x}\)
  \part \(g(x) = 3x^2+\frac{12}{\sqrt{x}}-\frac{1}{x^2}\)
  \end{parts}
  \begin{solution}
    \begin{enumerate}[(a)]
    \item \(12x^{11}\)
    \item \(\frac{3}{4}x^{-1/4}\)
    \item \(-\frac{3}{4}x^{-7/4}\)
    \item \(\ln e^{ax} = ax\) so \(d'(x) = a\)
    \item \(e(x) = x^{3/2}+x^{1/2}\) so \(e'(x) = \frac{3}{2}x^{1/2}+\frac{1}{2}x^{-1/2}\)
    \item \(f(x) = x+x^{-1}\) so \(f'(x) = 1-x^{-2}\)
    \item \(g(x) = 3x^2+12x^{-1/2}-x^{-2}\) so \(g'(x) = 6x-6x^{-3/2}+2x^{-3}\)
    \end{enumerate}
  \end{solution}
\question What is the derivative of \(f(x) = x^{\frac{1}{5}}\)? Is
  \(f\) differentiable at \(x=0\)?
  \begin{solution}
    \(f'(x) = \frac{1}{5}x^{-4/5}\). \(f'(0)\) is undefined since we
    can rewrite \(f'(x) = \frac{1}{5 x^{4/5}}\) and \(0^{4/5} = 0\) so
    we are dividing by \(0\). Geometrically, this is realized by
    noticing that the tangent line for \(x^{1/5}\) at \(x=0\) is vertical.
  \end{solution}
\question On what intervals is the graph of $g(x) = x^4-4x^3$ both decreasing and concave up?
  \begin{solution}
    We first compute \(g'(x) = 4x^3-12x^2\) and \(g''(x) =
    12x^2-24x\). Then, we check
    \begin{itemize}
    \item \(g(x)\) is decreasing \(\iff g'(x) < 0\). Therefore,
      \begin{align*}
        4x^3-12x^2 < 0
        & \implies 4x^2(x-3) < 0 \\
        & \implies x < 3
      \end{align*}
    \item \(g(x)\) is concave up \(\iff g''(x) > 0\). Therefore,
      \begin{align*}
        12x^2-24x > 0
        & \implies 12x(x-2) > 0 \\
        & \implies x > 2 \text{ or } x<0
      \end{align*}
    \end{itemize}
    Therefore, we have that \(g(x)\) is both decreasing and concave up
    on the intervals \((-\infty,0)\) and \((2,3)\).
  \end{solution}
\question For what values of $x$ is the function $f(x)=x^5-5x$ both increasing and concave up?	
  \begin{solution}
    We first compute \(f'(x) = 5x^4-5\) and \(f''(x) = 20x^3\). Then,
    we check
    \begin{itemize}
    \item \(f(x)\) is increasing \(\iff f'(x) > 0\). Therefore,
      \begin{align*}
        5x^4-5 > 0 & \implies 5x^4 > 5 \\
                   & \implies x^4 > 1 \\
        & \implies x < -1 \text{ or } x>1
      \end{align*}
    \item \(f(x)\) is concave up \(\iff f''(x) > 0\). Therefore,
      \begin{align*}
        20x^3 > 0 & \implies x^3 > 0\\
        & \implies x > 0
      \end{align*}
    \end{itemize}
    Thus, \(f(x)\) is both increasing and concave up on \((1,\infty)\).
  \end{solution}
\question The $n^{\text{th}}$ derivative of $f$, $f^{(n)}(x)$, is the result of differentiating $f(x)$ $n$ times. Consider the function $f(x) = x^7 + 5x^5 - 4x^3 + 6x - 7$.
\begin{enumerate}[(a)]
	\item Find the \emph{8th} derivative of $f(x)$. Think ahead!
	\item Find the 7th derivative of $f(x)$.
\end{enumerate}
\begin{solution}
  \begin{enumerate}[(a)]
  \item \(f^{(8)}(x) = 0\)
  \item \(f^{(7)}(x) = 7 \cdot 6 \cdot 5 \cdot 4 \cdot 3 \cdot 2 \cdot
    1\)
  \end{enumerate}
\end{solution}
        
\question
  \begin{parts}
  \part Find values for $a$ and $b$ so that the function $k$ is both
    continuous and differentiable everywhere.

$$k(x) = \left\{ \begin{array}{ll}
                   ax+2 & x<0\\
                   b(x-1)^2 & x\geq 0
                 \end{array}
               \right.$$
     \part What is \(k'(x)\)?
\end{parts}
\begin{solution}
  We know each piece of \(k(x)\) is continuous and differentiable
  everywhere, so we need only check at \(x=0\). 
  \begin{itemize}
  \item (Continuity) For this we need \(a \cdot 0 + 2 =
    b(0-1)^2\). This gives us \[
      b=2
    \]
  \item (Differentiability) For this, we need the derivative to be
    continuous. We know that \(k'(x) = a\) for \(x<0\) and \(k'(x) =
    2bx-2b = 4x-4\) for \(x > 0\). Thus, we need these two pieces to
    agree at \(x=0\). This gives us \[
      a = 4 \cdot 0 - 4 = -4
    \]
  \end{itemize}
  Thus, our final answer is \[
    k'(x) =
    \begin{cases}
      -4 & x < 0 \\
      4x-4 & x \geq 0
    \end{cases}
  \]
\end{solution}
\question Let \(p(x)\) be a seventh-degree polynomial with \(7\)
  distinct zeros. How many zeros does \(p'(x)\) have? Hint: use MVT
  to solve this.
  \begin{solution}
    If \(p(x)\) is a seventh-degree polynomial with \(7\) distinct
    zeros, this means the graph of \(p(x)\) crosses the \(x\)-axis
    \(7\) times. Consider any two consecutive zeros, say \(z_1\) and
    \(z_2\). Then, since \(p(x)\) is a polynomial, it is continuous on
    \([z_1,z_2]\) and differentiable on \((z_1,z_2)\). Thus, by the
    Mean Value Theorem, there exists a \(c\) satisfying \(z_1 < c <
    z_2\) such that \[
      p'(c) = \frac{p(z_2)-p(z_1)}{z_2-z_1} = \frac{0-0}{z_2-z_1} = 0
    \]
    Thus, since there are \(6\) pairs of consecutive zeros for
    \(p(x)\), we know \(p'(x)\) has \(6\) zeros.
  \end{solution}
\question At a time $t$ seconds after it is thrown up in the air, a tomato is at a height of $f(t)=-4.9 t^2 + 25 t + 3$ meters. 
\begin{enumerate}[(a)]
\item What is the average velocity of the tomato during the first $2$ seconds? Give units. 
\item Find (exactly) the instantaneous velocity of the tomato at $t=2$. Give units. 
\item What is the acceleration at $t=2$?
\item How high does the tomato go?
\item How long is the tomato in the air?
\end{enumerate}
\begin{solution}
  \begin{enumerate}[(a)]
  \item \[
\frac{f(2)-f(0)}{2-0} = \frac{-4.9\cdot 4 + 25 \cdot 2 + 3 - 3}{2-0} =
\frac{30.4}{2} = 15.2 \text{ meters/second}
    \]
  \item \(f'(t) = -9.8t+25\), so \(f'(2) = -19.6+25 = 5.4\) meters/second.
  \item \(f''(t) = -9.8\) so the acceleration at \(t=2\) is \(-9.8\) meters/(second)\({}^2\)
  \item If we graph \(f(t)\), we see it is a parabola with a max where
    the tangent line has slope \(0\). Thus, we can solve \[
      f'(t) = 0 \implies -9.8t+25 = 0 \implies t = \frac{25}{9.8}
      \text{ seconds}
    \]
    Thus, the maximum height is \(f(\frac{25}{9.8})\) meters.
  \item To solve this, we must check when \(f(t) = 0\) and pick the
    positive solution. \[
      f(t) = 0 \implies -4.9t^2+25t+3 = 0 \implies t = \frac{-25
      \pm \sqrt{25^2-4(-4.9)(3)}}{-9.8} = \frac{25 \pm \sqrt{525 \pm 58.8}}{9.8}
    \]
    Thus, we pick the positive solution \(t =
    \frac{25+\sqrt{583.8}}{9.8}\) seconds. (This is about 5.02 seconds.)
  \end{enumerate}
\end{solution}
\pagebreak
\question
  \begin{enumerate}[(a)]
  \item Find an equation of the line tangent to the graph of
    \(f(x) = \sqrt{x}\) at the point \((4,2)\) on the graph.
  \item For \(f(x) = \sqrt{x}\), what is \(\lim_{x\to \infty} f'(x)\)?
    How is this consistent with the graph of \(f(x)\)?
  \end{enumerate}
  \begin{solution}
    We first solve \(f'(x) = \frac{1}{2}x^{-1/2}\). Thus, the tangent
    line has slope \(f'(4) = \frac{1}{2} (4)^{-1/2} = \frac{1}{4}\).
    Using point-slope form, the tangent line is
    given by \[
      y = \frac{1}{4}(x-4)+2
    \]
    We compute that \(\lim_{x \to \infty} \frac{1}{2}x^{-1/2} =
    0\). This is okay because \(f'(x)\) is positive for \(x > 0\) but
    decreasing. 
  \end{solution}
\question (Fall 2018 Exam 2) Let \(A\) and \(B\) be constants and \[
    k(x) =
    \begin{cases}
      3x+\frac{B}{x} & \text{for }0 < x < 1\\
      Bx^2+Ax^3 & \text{for } 1 \leq x
    \end{cases}
  \]
  Find the values of \(A\) and \(B\) that make the function \(k(x)\) differentiable on \((0, \infty)\). Show all your
work to justify your answers. If there are no such values of \(A\) and \(B\), write none.
\begin{solution}
  See \href{https://dhsp.math.lsa.umich.edu/exams/115exam2/f18/s2.pdf}{https://dhsp.math.lsa.umich.edu/exams/115exam2/f18/s2.pdf}
\end{solution}
\question Suppose $p$ is a cubic polynomial function, meaning that $p(x) = a_3x^3 +a_2x^2 +a_1x+a_0$
for some constants $a_0, a_1, a_2, a_3$, with $a_0 \neq 0$.
\begin{enumerate}[(a)]
\item Write expressions for $p(0)$, $p'(0)$, $p''(0)$ and $p'''(0)$ depending on $a_0, a_1, a_2$, and $a_3$.
\item Find the formula for a cubic polynomial function $q$ that satisfies
\[
q(0) = 2, \quad q'(0) = -1, \quad q''(0) = 5, \quad q'''(0) = 4.
\]
\end{enumerate}
\begin{solution}
  Using our power rule for differentiating a polynomial, we can check
  \begin{itemize}
  \item \(p(0) = a_0\)
  \item \(p'(x) = 3 a_3 x^2 + 2 a_2 x + a_1 \implies p'(0) = a_1\) 
  \item \(p''(x) = 6 a_3 x + 2 a_2 \implies p''(0) = 2 a_2\)
  \item \(p'''(x) = 6 a_3 \implies p'''(0) = 6 a_3\)
  \end{itemize}
  Thus, to find a cubic polynomial \(q\) satisfying the above
  identities, we check
  \begin{itemize}
  \item \(a_0 = q(0) = 2\)
  \item \(a_1 = q'(0) = -1\)
  \item \(2 a_2 = q''(0) = 5 \implies a_2 = \frac{5}{2}\)
  \item \(6 a_3 = q'''(0) = 4 \implies a_3 = \frac{2}{3}\)
  \end{itemize}
  Thus, \(q(x) = \frac{2}{3}x^3+\frac{5}{2}x^2-x+2\) satisfies the
  desired properties.
\end{solution}
\question Assume that $f''$ and $g''$ exist and that $f$ and $g$ are concave up for all $x$. Are the following statements true or false? If a statement is true, explain how you know. If a statement is false, give a counterexample.
\begin{enumerate}[(a)]
	\item $f(x)+g(x)$ is concave up for all $x$.
	\item $f(x)-g(x)$ cannot be concave up for all $x$.
\end{enumerate}
\begin{solution}
  \begin{enumerate}[(a)]
  \item True. The second derivative of \(h(x) = f(x)+g(x)\) is
    \(h''(x) = f''(x)+g''(x)\) and both \(f''(x) > 0\) and \(g''(x) >
    0\), so \(f''(x)+g''(x) = h''(x) > 0\).
  \item False. It could be that \(f''(x) > g''(x)\) for all \(x\).
  \end{enumerate}
\end{solution}
\question Let $f(x) = x^4 - 3x^2 + 1$.
\begin{enumerate}[(a)]
	\item Show that $f(x)$ is an even function.
	\item Show that $f'(x)$ is an odd function.
	\item Are all polynomials of even degree even functions?
\end{enumerate}
\begin{solution}
  \begin{enumerate}[(a)]
  \item \(f(-x) = (-x)^4-3(-x)^2+1 = x^4-3x^2+1 = f(x)\) 
  \item \(f'(x) = 4 x^3 - 6 x\). Then, \(f'(-x) = 4(-x)^3-6(-x) =
    -4x^3+6x = -f'(x)\)
  \item No. Every monomial must be of even degree.
  \end{enumerate}
\end{solution}
\question (Winter 2016 Exam 3) For constants $A$ and $B$, consider the function $h$ defined by
\[
h(t) = \begin{cases}
(At)^2-48	& \text{if } t < 2\\
Bt^3		& \text{if } t \geqslant 2.	
\end{cases}
\]
Circle \underline{all} pairs of values of $A$ and $B$ such that $h(t)$ is differentiable.
\begin{multicols}{3}
    \begin{enumerate}[i.]
        \item $A = 3, B = 3$
        \item $A = 6, B = 12	$
        \item $A = -6, B = 12$
        \item $A = 0, B = 0$
        \item $A = 18, B = 12$
        \item \textsc{none of these}
    \end{enumerate}
\end{multicols}
\begin{solution}
  See part e in \href{https://dhsp.math.lsa.umich.edu/exams/115exam3/w16/s10.pdf}{https://dhsp.math.lsa.umich.edu/exams/115exam3/w16/s10.pdf}
\end{solution}
\end{questions}
\end{document}
%%% Local Variables:
%%% mode: latex
%%% TeX-master: t
%%% End:
