\documentclass[11pt]{exam}
\usepackage[margin=1in]{geometry}
\pagestyle{plain}
\usepackage{amsmath,amsfonts,amssymb,amsthm,enumerate}
\usepackage{multicol}
\usepackage[]{graphicx}
\usepackage{hyperref}

\addtolength{\footskip}{2\baselineskip} % to lower the page numbers
\title{\vspace{-0.5in} Math 115 \\ Worksheet Section 1.2}
\date{}
%\printanswers

% \theoremstyle{definition}
% \newtheorem{problem}{Problem}
% \qformat{Problem \thequestion. \hfill} 
\renewcommand{\questionlabel}{\textbf{Problem~\thequestion.}}

\begin{document}
\maketitle
\vspace{-0.75in}
\begin{questions}
 \question (Warm-up). We say $P$ is an exponential function of $t$
 with base $a$ if $P(t) =$\fillin[\(P_0 a^t\)].

If \fillin[\(a > 1\)][1.5cm] then we have exponential growth; if \fillin[\(0<a<1\)] then we have exponential decay.

Fill in the following table for \(P(t)\). \vskip.4ex
\begin{tabular}{c|c}
\(t\) & \(P(t)\) \\
\hline
0 & \\
1 & \\
2\\
3\\
\end{tabular}
\begin{solution}
 \begin{tabular}{c|c}
\(t\) & \(P(t)\) \\
\hline
0 & \(P_0\) \\
1 & \(P_0 a\)\\
2 & \(P_0 a^2\)\\
3 & \(P_0 a^3\)\\
\end{tabular}
 
\end{solution}
\vskip.4ex
Notice anything interesting about \(P(1)/P(0)\), \(P(2)/P(1)\), and
\(P(3)/P(2)\)?
\begin{solution}
  They all are equal to \(a\).
\end{solution}
\vskip1ex
We can rewrite this exponential function with base $e$, in the form
\fillin[\(P(t) = P_0 e^{kt}\)].

If \fillin[\(k>0\)][1.5cm] then we have exponential growth; if \fillin[\(k<0\)] then we have exponential decay.

\vskip0.5cm
\question Newton's law of heating says:
\begin{itemize}
	\item{
		The difference between the temperature $\theta$ of an object at time $t$ and the temperature $\theta_s$ of its surroundings is an exponential function $H(t)$.
	}
	\item{
		$H(t)$ has initial quantity the difference between the initial temperature $\theta_0$ of the object and the temperature of its surroundings.
	}
	\item{
		$H(t)$ decays at continuous rate equal to the material
                constant $k$ of the object. 
	}
\end{itemize}
It is assumed that the surroundings of the object have constant temperature.
\begin{parts}
  \part Find an expression for $H(t)$. 
  \begin{solution}
     \(H(t) = (\theta_0-\theta_s)e^{-kt} \)
  \end{solution}
  \vspace{0.5in}
  The one ring has initial
  temperature $20^\circ$C when it is thrown into the molten core of
  Mount Doom. the one ring melts after $2$ seconds. The one ring was
  made from a gold-silver alloy with a melting point of $1015^\circ$C
  and material constant $k=\ln(2/\sqrt{3})$.
  \part Calculate the temperature of Mount Doom. 
  \begin{solution}
    We wish to know \(\theta_s\), the surrounding temperature of Mount
    Doom. Let \(\theta(t)\) be the temperature of the one ring. We can use
    Newton's law of heating to compute \(\theta_s\) since we know
    \begin{itemize}
    \item \(\theta(t) = \theta_s+H(t)\),
    \item \(\theta_0 = 20^\circ\)C,
    \item \(\theta(2) = 1015^\circ\)C,
    \item \(k = \ln(2/\sqrt{3})\).
    \end{itemize}
    We combine this to get, \[
      \begin{cases}
        H(2) = \theta(2)-\theta_s\\
        H(2) = (20-\theta_s)e^{-\ln(2/\sqrt{3})\cdot 2}
      \end{cases}
    \]
    Then, we combine these equations to get
    \begin{align*}
      1015-\theta_s = (20-\theta_s)e^{-\ln(2/\sqrt{3})\cdot 2}
      & \implies 1015-\theta_s = (20-\theta_s) \left(
        \frac{\sqrt{3}}{2} \right)^2\\
      & \implies 1015-\theta_s = (20-\theta_s) \frac{3}{4}\\
      & \implies 1015-\theta_s = 15-\frac{3}{4}\theta_s\\
      & \implies 1000 = \frac{1}{4}\theta_s\\
      & \implies \theta_s = 4000
    \end{align*}
    Thus, Mount Doom is \(4000^\circ\)C.
  \end{solution}
\end{parts}
\pagebreak
\question (Fall 2011 Exam 1 Problem 4) 	A zombie plague has broken out in Ann Arbor. As a nurse in the University Hospital, you saw the person with the first case of the plague, patient zero.
\begin{parts}
 \part In order to keep track of the growing zombie population in Ann Arbor, you collected the following data: 
\begin{center}
\begin{tabular}{ | c | c | c | c | c | c | c | c | }
\hline 
 Days after patient zero & 0 & 6 & 9 & 12  \\ 
 \hline
 Number of zombies & 1 & 9 & 27 & 81 \\  
 \hline
 \end{tabular}
\end{center}
Would a linear function or an exponential function be the best model? Why?
\part Write a function \(Z(t)\) of the appropriate type to model the growth of the zombie population at time $t$ measured in days after patient zero.
\part The population of North America is approximately 530,000,000 people. Using your model, how long will it take until all but one person are infected?
\end{parts}
\begin{solution}
  See \href{https://dhsp.math.lsa.umich.edu/exams/115exam1/f11/s3.pdf}{https://dhsp.math.lsa.umich.edu/exams/115exam1/f11/s3.pdf}
\end{solution}
\vspace{1in}
\question (Winter 2018 Exam 1 Problem 9)
A new video is released and a few hours later it goes viral. The number of views, in thousands, of the video \(t\) hours after it goes viral is given by the function $v(t)$. For the first 24 hours, the number of views of the video is increasing exponentially, reaching 50,000 views 12 hours after going viral and 120,000 views 24 hours after going viral. After that, during the second 24 hours, the video is gaining 10,000 views every 3 hours.
\begin{enumerate}[(a)]
\item Find a piecewise defined formula for $v(t)$ for $0 \leqslant t \leqslant 48$. Show all your work.

\item Find the hourly percentage growth rate of \(v(t)\) during the
  first 24 hours. Give only an exact form of your answer.
\end{enumerate}
\begin{solution}
  See \href{https://dhsp.math.lsa.umich.edu/exams/115exam1/w18/s9.pdf}{https://dhsp.math.lsa.umich.edu/exams/115exam1/w18/s9.pdf}
\end{solution}
\vspace{1in}
\question (Fall 2017 Exam 1 Problem 7)
After testing different ingredients in their parents’ garages, Imran and Nicole have
recently opened new organic peanut butter companies.
\begin{parts}
\part Two months after opening, Imran’s company, Chunky Munky, has
  produced a total of 256 pounds of peanut butter. Imran thinks Chunky
  Munky produces peanut butter at a constant rate of 690 pounds every
  6 months. Assuming Imran is correct, write a formula for P(m), the
  total amount of peanut butter, in pounds, that Chunky Munky will
  have produced m months after opening.

\part Nicole’s company, Lots O’ Crunch, has produced a total of 182
  pounds of peanut butter two months after opening and a total 454
  pounds of peanut butter five months after opening. Nicole thinks
  that Lots O’ Crunch produces peanut butter exponentially. Assuming
  Nicole is correct, write a formula for \(Q(x)\), the total amount of
  peanut butter, in pounds, Lots O’ Crunch will have produced \(x\) months
  after opening.
\end{parts}
\begin{solution}
 See \href{https://dhsp.math.lsa.umich.edu/exams/115exam1/f17/s7.pdf}{https://dhsp.math.lsa.umich.edu/exams/115exam1/f17/s7.pdf}
\end{solution}
\end{questions}
\end{document}

%%% Local Variables:
%%% mode: latex
%%% TeX-master: t
%%% End:
